Il est désormais important de remettre en perspective le travail de mise en relation réalisé ici. Ce travail s’inscrit dans une dynamique plus large d’ouverture des données de la recherche (\index{Open Data}Open Data), en contribuant à la fois à la collaboration scientifique, à la valorisation des données, et à la facilitation de leur accès pour les utilisateurs\footcite{wesselsOpenDataKnowledge2017}.


\section{Renforcement des collaborations scientifiques}
La connexion entre les deux bases de données a non seulement permis de relier des données éloignées, mais aussi de rapprocher des équipes de recherche géographiquement et institutionnellement distinctes\footnote{Je me permets de renvoyer à l'introduction pour plus de détails}.

Un rapprochement s’est opéré, notamment au niveau institutionnel. Au cours de ce stage, des réunions ont parfois été nécessaires entre les équipes des deux bases. Ces réunions avaient pour objectif non seulement de présenter l’avancement de mon travail, mais aussi de réfléchir collectivement aux défis rencontrés. Par exemple, une réunion avec \index{Marie-Anne Polo de Beaulieu}Marie-Anne Polo de Beaulieu, Elisa Lonati et \index{Isabelle Draelants}Isabelle Draelants a permis de tenir l’équipe de ThEMA informée de mes progrès alors que je collaborais avec l’équipe de SourcEncyMe. Ce moment a été crucial pour déterminer comment intégrer les \index{Récits exemplaires}récits exemplaires trouvés dans les encyclopédies au sein de ThEMA. Par ailleurs, chaque équipe a eu l’occasion de présenter son travail à l'autre avant le début du stage.

Ce rapprochement s’est également opéré au niveau des données, car j’ai réussi à connecter les deux bases qui, auparavant, étaient indépendantes. Cela a permis d’éviter la duplication des efforts et de favoriser la réutilisation des données, tout en évitant que SourcEncyMe ne devienne une base spécialisée dans les exempla, et vice versa.


\section{Valorisation des bases de données}
La mise en relation des deux bases de données a également permis de valoriser chacune d’elles.

Ce stage m’a permis d’ajouter de la valeur à ces deux bases. Pour SourcEncyMe, j’ai contribué à l’identification des \index{Récits exemplaires}récits exemplaires dans les encyclopédies, qui n’avaient pas été repérés auparavant. Ce processus pourra d’ailleurs se poursuivre grâce au code \index{XSL}XSL que j’ai développé. Pour ThEMA, j’ai enrichi la base avec de nouveaux \textit{exempla} issus des encyclopédies, notamment en traitant intégralement deux œuvres : le \index{Speculum historiale}\textit{Speculum historiale} de Vincent de Beauvais et la \textit{Summa de exemplis ac similitudinibus rerum} de Giovanni da San Gimignano. Il reste toutefois à télécharger les fichiers \index{XML}XML modifiés dans les deux bases pour rendre visible l’ensemble du travail réalisé durant mon stage\footnote{Les fichiers ont été envoyés aux deux équipes, mais restent disponibles sur mon GitHub. \url{https://github.com/Laitauchocolat34/memoire-tnah}}. Par ailleurs, les codes en annexes permettront de traiter toutes les autres encyclopédies de SourcEncyMe que je n'ai pas eu le temps de traiter pendant mon stage.

Ce stage a également contribué à rendre les deux bases de données plus accessibles et plus visibles. Les liens ajoutés augmentent la probabilité que les utilisateurs découvrent l’une des bases. J’ai aussi présenté, avec \index{Marie-Anne Polo de Beaulieu}Marie-Anne Polo de Beaulieu, le travail en cours sur les deux bases de données au collectif « Sources et données de la recherche » du \index{CRH}CRH, et un billet est prévu sur le blog de l’atelier de \index{Vincent de Beauvais}Vincent de Beauvais pour mettre en lumière ce travail. Par ailleurs, il est également envisagé d'informer le LabEx HaStec des avancées du travail.


\section{Amélioration de l'expérience utilisateur}
Enfin, la mise en relation des deux bases de données a facilité la recherche pour les utilisateurs. Auparavant, s’ils trouvaient un texte encyclopédique dans ThEMA ou un \textit{exemplum} dans SourcEncyMe, ils ne savaient peut-être pas où chercher pour obtenir plus d’informations. Désormais, un utilisateur de SourcEncyMe qui trouve un \textit{exemplum} peut consulter la bibliographie associée, découvrir d’autres \index{Récits exemplaires}récits exemplaires sur les mêmes thèmes, ou encore accéder à des textes similaires. De son côté, l’utilisateur de ThEMA pourra plus facilement retrouver le texte intégral de l’encyclopédie et identifier la source de l’auteur de l'encyclopédie. 

Cependant, pour l’utilisateur, il s’agit souvent d’un « gain marginal » dans ce type de bases de données. En général, l’utilisateur explore une base pour une raison précise et ne cherche pas nécessairement à exploiter toutes les possibilités offertes. Globalement, l’accessibilité en ligne de la base de données reste l’aspect le plus crucial, tandis que le reste peut être perçu comme un complément. Pour un utilisateur, la création et la mise en ligne d'une base de données sont plus importantes que l'établissement de liens entre différentes bases.