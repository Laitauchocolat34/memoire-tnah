\begin{lstlisting}[breaklines=true]
	import os
	import xml.etree.ElementTree as ET
	import pandas as pd
	import re
	
	# Définir l'espace de noms TEI
	TEI_NAMESPACE = 'http://www.tei-c.org/ns/1.0'
	NS_MAP = {'tei': TEI_NAMESPACE}
	ET.register_namespace('', TEI_NAMESPACE)  # Enregistrer l'espace de noms par défaut
	
	def roman_to_arabic(roman):
		"""
		Convertir les chiffres romains en chiffres arabes.
		"""
		roman_numerals = {'I': 1, 'V': 5, 'X': 10, 'L': 50, 'C': 100, 'D': 500, 'M': 1000}
		arabic = 0
		prev_value = 0
		for char in reversed(roman):
			value = roman_numerals.get(char, 0)
			if value < prev_value:
				arabic -= value
			else:
				arabic += value
			prev_value = value
		return arabic
	
	def extract_numbers_from_text(text, search_term):
		"""
		Extraire les numéros à partir du texte en utilisant l'expression régulière spécifiée.
		Convertir les chiffres romains en chiffres arabes et ajouter 1 au premier numéro.
		"""
		numbers = {'numero1': '', 'numero2': ''}
		match = re.search(search_term, text)
		if match:
			num1 = match.group(1) if match.group(1) else ''
			num2 = match.group(2) if match.group(2) else ''
	
			# Convertir le premier numéro en chiffres arabes s'il est en chiffres romains
			if num1 and re.match(r'^[IVXLCDM]+$', num1):
				num1_arabic = roman_to_arabic(num1)
				numbers['numero1'] = str(num1_arabic + 1)  # Ajouter 1 à num1
			elif num1.isdigit():
				numbers['numero1'] = str(int(num1) + 1)  # Ajouter 1 à num1 s'il est déjà en chiffres arabes
			else:
				numbers['numero1'] = num1
	
			numbers['numero2'] = num2
		return numbers
	
	def check_numbers_in_other_file(numbers, other_xml_root, base_xml_id):
		"""
		Vérifier si les numéros extraits existent dans le fichier XML autre.
		Retourner True et l'ID de la première citation correspondante si trouvée, sinon False.
		"""
		numero1 = numbers['numero1']
		numero2 = numbers['numero2']
		base_xml_url = f"https://thema.huma-num.fr/exempla/{base_xml_id}"
	
		for div in other_xml_root.findall('.//tei:div', NS_MAP):
			xml_id = div.attrib.get('{http://www.w3.org/XML/1998/namespace}id', '')
			if xml_id:
				parts = xml_id.split('.')
				if len(parts) >= 3:
					div_numero1 = parts[1]
					div_numero2 = parts[2]
					if (numero1 == div_numero1 and (not numero2 or numero2 == div_numero2)):
						first_cit = div.find('.//tei:cit', NS_MAP)
						if first_cit is not None:
							cit_id = first_cit.attrib.get('{http://www.w3.org/XML/1998/namespace}id', '')
							if 'corresp' not in div.attrib:
								div.set('corresp', base_xml_url)
							else:
								div.attrib['corresp'] = base_xml_url
							return True, cit_id
			return False, None
	
	def add_cit_id_to_file(file_path, cit_id):
		"""
		Ajouter un élément <item> avec le cit_id dans la section <list type="links"> d'un fichier XML.
		"""
		try:
			tree = ET.parse(file_path)
			xml_root = tree.getroot()
		
			# Trouver ou créer la section <list type="links">
			list_links = xml_root.find('.//tei:list[@type="links"]', NS_MAP)
			if list_links is None:
				# Créer <list type="links"> si elle n'existe pas
				source_desc = xml_root.find('.//tei:sourceDesc', NS_MAP)
				if source_desc is not None:
					list_links = ET.SubElement(source_desc, '{http://www.tei-c.org/ns/1.0}list')
					list_links.set('type', 'links')
		
			if list_links is not None:
				# Ajouter le nouvel élément <item>
				new_item = ET.SubElement(list_links, '{http://www.tei-c.org/ns/1.0}item')
				new_item.set('type', 'link')
				new_item.set('corresp', f"http://sourcencyme.irht.cnrs.fr/encyclopedie/speculum_naturale_version_sm_trifaria_ed_douai_1624?citid={cit_id}")
				new_item.text = "SourcEncyMe"
		
			# Enregistrer les modifications
			tree.write(file_path, encoding='utf-8', xml_declaration=True)
		except (ET.ParseError, Exception) as e:
			print(f"Erreur lors de la modification du fichier {file_path}: {e}")
	
	def search_in_tei_files(directory_path, search_term, other_file_path):
		"""
		Rechercher le terme spécifié dans les fichiers TEI XML d'un répertoire et vérifier s'il correspond
		dans un autre fichier TEI XML.
		Retourner une liste des paragraphes correspondants avec les détails.
		"""
		matching_paragraphs = []
	
		# Vérifier si le répertoire existe
		if not os.path.exists(directory_path):
			print(f"Le répertoire {directory_path} n'existe pas.")
			return matching_paragraphs
	
		# Charger le contenu du fichier XML autre
		try:
			with open(other_file_path, 'r', encoding='utf-8') as f:
				other_tree = ET.ElementTree(ET.fromstring(f.read()))
			other_xml_root = other_tree.getroot()
		except (FileNotFoundError, ET.ParseError, Exception) as e:
			print(f"Erreur avec le fichier {other_file_path}: {e}")
			return matching_paragraphs
	
	# Parcourir les fichiers dans le répertoire spécifié
	for root, _, files in os.walk(directory_path):
		for file_name in files:
			if file_name.endswith('.xml'):
				file_path = os.path.join(root, file_name)
				try:
					tree = ET.parse(file_path)
					xml_root = tree.getroot()
					items = xml_root.findall('.//tei:item[@type="sources"]/tei:p', NS_MAP)
					base_xml_id = xml_root.attrib.get('{http://www.w3.org/XML/1998/namespace}id', '')
	
					for item in items:
						if item.text:
							numbers = extract_numbers_from_text(item.text, search_term) 
							match, cit_id = check_numbers_in_other_file(numbers, other_xml_root, base_xml_id)
							if match:
								matching_paragraphs.append({
									'file_path': file_path, 
									'paragraph': item.text.strip(),
									'numero1': numbers['numero1'],
									'numero2': numbers['numero2'],
									'cit_id': cit_id,
									'base_xml_id': base_xml_id
								})
								print(f"Correspondance trouvée :\nFichier : {file_path}\nParagraphe : {item.text.strip()}\nNuméro 1 : {numbers['numero1']}\nNuméro 2 : {numbers['numero2']}\nCIT ID : {cit_id}\nBase XML ID : {base_xml_id}\n")
								# Ajouter le cit_id au fichier
								add_cit_id_to_file(file_path, cit_id)
	
				except (ET.ParseError, Exception) as e:
					print(f"Erreur avec le fichier {file_path}: {e}")
	
		# Enregistrer le fichier modifié other_file_path
		other_tree.write(other_file_path, encoding='utf-8', xml_declaration=True)
	
		return matching_paragraphs
	
	# Exemple d'utilisation :
	directory_path = 'data'  # Assurez-vous que ce chemin est correct
	search_term = r'S?s?peculum historiale,? ([IVXLCDM]+|\d+)[.,]?\s*([0-9]+)?'  # Terme à rechercher avec regex
	other_file_path = 'vincentius_belvacensis-speculum_historiale_version_sm_trifaria_ms_douai_bm_797.xml'  # Chemin vers l'autre fichier XML TEI à vérifier
	
	matching_paragraphs = search_in_tei_files(directory_path, search_term, other_file_path)
	
	# Créer un DataFrame pandas pour les résultats
	df = pd.DataFrame(matching_paragraphs)
	
	# Exporter les résultats dans un fichier Excel
	output_file = 'resultats_recherche.xlsx'
	df.to_excel(output_file, index=False)
	
	print(f"Les résultats de la recherche ont été exportés vers {output_file}.")
	
\end{lstlisting}